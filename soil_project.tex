\documentclass[fleqn,]{article} % Document font size and equations flushed left

\setcounter{tocdepth}{3}

% Pandoc environments
\usepackage{framed}
\usepackage{fancyvrb}
\providecommand{\tightlist}{%
  \setlength{\itemsep}{0pt}\setlength{\parskip}{0pt}}
\newcommand{\VerbBar}{|}
\newcommand{\VERB}{\Verb[commandchars=\\\{\}]}
\DefineVerbatimEnvironment{Highlighting}{Verbatim}{commandchars=\\\{\}, fontsize=\scriptsize} % R Code

% Colored code
\usepackage{color}
\definecolor{shadecolor}{RGB}{248,248,248}
\newenvironment{Shaded}{\begin{snugshade}}{\end{snugshade}}
\newcommand{\KeywordTok}[1]{\textcolor[rgb]{0.13,0.29,0.53}{\textbf{{#1}}}}
\newcommand{\DataTypeTok}[1]{\textcolor[rgb]{0.13,0.29,0.53}{{#1}}}
\newcommand{\DecValTok}[1]{\textcolor[rgb]{0.00,0.00,0.81}{{#1}}}
\newcommand{\BaseNTok}[1]{\textcolor[rgb]{0.00,0.00,0.81}{{#1}}}
\newcommand{\FloatTok}[1]{\textcolor[rgb]{0.00,0.00,0.81}{{#1}}}
\newcommand{\ConstantTok}[1]{\textcolor[rgb]{0.00,0.00,0.00}{{#1}}}
\newcommand{\CharTok}[1]{\textcolor[rgb]{0.31,0.60,0.02}{{#1}}}
\newcommand{\SpecialCharTok}[1]{\textcolor[rgb]{0.00,0.00,0.00}{{#1}}}
\newcommand{\StringTok}[1]{\textcolor[rgb]{0.31,0.60,0.02}{{#1}}}
\newcommand{\VerbatimStringTok}[1]{\textcolor[rgb]{0.31,0.60,0.02}{{#1}}}
\newcommand{\SpecialStringTok}[1]{\textcolor[rgb]{0.31,0.60,0.02}{{#1}}}
\newcommand{\ImportTok}[1]{{#1}}
\newcommand{\CommentTok}[1]{\textcolor[rgb]{0.56,0.35,0.01}{\textit{{#1}}}}
\newcommand{\DocumentationTok}[1]{\textcolor[rgb]{0.56,0.35,0.01}{\textbf{\textit{{#1}}}}}
\newcommand{\AnnotationTok}[1]{\textcolor[rgb]{0.56,0.35,0.01}{\textbf{\textit{{#1}}}}}
\newcommand{\CommentVarTok}[1]{\textcolor[rgb]{0.56,0.35,0.01}{\textbf{\textit{{#1}}}}}
\newcommand{\OtherTok}[1]{\textcolor[rgb]{0.56,0.35,0.01}{{#1}}}
\newcommand{\FunctionTok}[1]{\textcolor[rgb]{0.00,0.00,0.00}{{#1}}}
\newcommand{\VariableTok}[1]{\textcolor[rgb]{0.00,0.00,0.00}{{#1}}}
\newcommand{\ControlFlowTok}[1]{\textcolor[rgb]{0.13,0.29,0.53}{\textbf{{#1}}}}
\newcommand{\OperatorTok}[1]{\textcolor[rgb]{0.81,0.36,0.00}{\textbf{{#1}}}}
\newcommand{\BuiltInTok}[1]{{#1}}
\newcommand{\ExtensionTok}[1]{{#1}}
\newcommand{\PreprocessorTok}[1]{\textcolor[rgb]{0.56,0.35,0.01}{\textit{{#1}}}}
\newcommand{\AttributeTok}[1]{\textcolor[rgb]{0.77,0.63,0.00}{{#1}}}
\newcommand{\RegionMarkerTok}[1]{{#1}}
\newcommand{\InformationTok}[1]{\textcolor[rgb]{0.56,0.35,0.01}{\textbf{\textit{{#1}}}}}
\newcommand{\WarningTok}[1]{\textcolor[rgb]{0.56,0.35,0.01}{\textbf{\textit{{#1}}}}}
\newcommand{\AlertTok}[1]{\textcolor[rgb]{0.94,0.16,0.16}{{#1}}}
\newcommand{\ErrorTok}[1]{\textcolor[rgb]{0.64,0.00,0.00}{\textbf{{#1}}}}
\newcommand{\NormalTok}[1]{{#1}}

% cslreferences environment required by pandoc > 2.7

% Polyglossia
\usepackage{polyglossia}
\setmainlanguage{en-UK}
\setotherlanguage{fr-FR}
\setotherlanguage{it}

% localized quotes
\usepackage[strict,autostyle]{csquotes}

% Figures
\usepackage{graphicx,grffile}
\makeatletter
\def\maxwidth{\ifdim\Gin@nat@width>\linewidth\linewidth\else\Gin@nat@width\fi}
\def\maxheight{\ifdim\Gin@nat@height>\textheight0.8\textheight\else\Gin@nat@height\fi}
\makeatother
% Scale images if necessary, so that they will not overflow the page
% margins by default, and it is still possible to overwrite the defaults
% using explicit options in \includegraphics[width, height, ...]{}
\setkeys{Gin}{width=\maxwidth,height=\maxheight,keepaspectratio}

% Additional packages
\usepackage {natbib}             % Advanced Bibliography (citep...).
\usepackage {amsmath,amsfonts,amssymb}
\usepackage {breqn}              % Line breaks in equations
\usepackage {url}                % Line breaks in url's
\usepackage {enumitem}           % Line spacing in lists
  \setlist[itemize]{noitemsep,nolistsep}
  \setlist[enumerate]{noitemsep,nolistsep}

% Tables
\usepackage{longtable,booktabs,tabu}
\usepackage{caption}                % Après babel
% These lines are needed to make table captions work with longtable:
\makeatletter
\def\fnum@table{\tablename~\thetable}
\makeatother
\usepackage{tabularx}               % Line breaks in tables
  \renewcommand{\arraystretch}{1.8}
\usepackage{multirow}               % Merged lines in tables

% Prevent overfull lines
\setlength{\emergencystretch}{3em}

% User-adder preamble
\hyphenation{bio-di-ver-si-ty sap-lings}

% hyperref comes last
\usepackage {hyperref}           % Hypertext links, PDF bookmarks
  \hypersetup{%
    urlcolor=blue,%
    linkcolor=black,citecolor=black,colorlinks=true%
  }



%----------------------------------------------------------------------------------------
%	ARTICLE INFORMATION
%----------------------------------------------------------------------------------------

\title{Title of the Article} % Article title

\author{
Hector Martin\\ Pierrot Froment\\ AUdrey Heyraud
} % Authors


\begin{document}

\selectlanguage{en-UK}

\maketitle % Print the title and abstract box

\begin{abstract}
Abstract of the article.
\end{abstract}


\thispagestyle{empty} % Removes page numbering from the first page

%----------------------------------------------------------------------------------------
%	ARTICLE CONTENTS
%----------------------------------------------------------------------------------------

\section{Introduction}\label{introduction}

\section{Method}\label{method}

\subsection{Study site}\label{study-site}

text, a document in different formats: HTML, LaTeX or Word for example.

\subsection{Sampling design}\label{sampling-design}

In RStudio, create a new document of type Document R Markdown. The wizard allows you to choose between different formats.

\section{Results}\label{results}

The main features of Markdown are summarized here.

\subsection{Wood decomposition}\label{wood-decomposition}

R code is included in code chunks:

\subsection{Tables}\label{tables}

\scriptsize

\begin{Shaded}
\begin{Highlighting}[]
\FunctionTok{names}\NormalTok{(iris) }\OtherTok{\textless{}{-}} \FunctionTok{c}\NormalTok{(}\StringTok{"Sepal length"}\NormalTok{, }\StringTok{"Width"}\NormalTok{, }\StringTok{"Petal length"}\NormalTok{,}
    \StringTok{"Width"}\NormalTok{, }\StringTok{"Species"}\NormalTok{)}
\NormalTok{kableExtra}\SpecialCharTok{::}\FunctionTok{kbl}\NormalTok{(}\FunctionTok{head}\NormalTok{(iris), }\AttributeTok{caption =} \StringTok{"Table created by R"}\NormalTok{,}
    \AttributeTok{longtable =} \ConstantTok{TRUE}\NormalTok{, }\AttributeTok{booktabs =} \ConstantTok{TRUE}\NormalTok{) }\SpecialCharTok{\%\textgreater{}\%}
\NormalTok{    kableExtra}\SpecialCharTok{::}\FunctionTok{kable\_styling}\NormalTok{(}\AttributeTok{bootstrap\_options =} \StringTok{"striped"}\NormalTok{,}
        \AttributeTok{full\_width =} \ConstantTok{FALSE}\NormalTok{)}
\end{Highlighting}
\end{Shaded}

\begin{longtable}[t]{rrrrl}
\caption{\label{tab:kable}Table created by R}\\
\toprule
Sepal length & Width & Petal length & Width & Species\\
\midrule
5.1 & 3.5 & 1.4 & 0.2 & setosa\\
4.9 & 3.0 & 1.4 & 0.2 & setosa\\
4.7 & 3.2 & 1.3 & 0.2 & setosa\\
4.6 & 3.1 & 1.5 & 0.2 & setosa\\
5.0 & 3.6 & 1.4 & 0.2 & setosa\\
\addlinespace
5.4 & 3.9 & 1.7 & 0.4 & setosa\\
\bottomrule
\end{longtable}

\normalsize

\subsection{Community composition}\label{community-composition}

\scriptsize

\begin{Shaded}
\begin{Highlighting}[]
\FunctionTok{plot}\NormalTok{(pressure)}
\end{Highlighting}
\end{Shaded}

\begin{figure}

{\centering \includegraphics[width=0.8\linewidth]{soil_project_files/figure-latex/pressure-1} 

}

\caption{Figure title}\label{fig:pressure}
\end{figure}

\normalsize

\subsection{PDF Document}\label{pdf-document}

The document is formatted according to the article LaTeX template.

%----------------------------------------------------------------------------------------
%	REFERENCE LIST
%----------------------------------------------------------------------------------------

\bibliographystyle{chicago}
\makeatletter
% The filename has .bib extension that must be eliminated
\filename@parse{references.bib}
% parse stores the file name in base. Extension starts at the first dot, so don't use dots in file names.
\bibliography{\filename@base}
\makeatother


%----------------------------------------------------------------------------------------

\end{document}
